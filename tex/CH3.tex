\newpage % Rozdziały zaczynamy od nowej strony.
\section{Wybrane algorytmy uczenia maszynowego}
 przedstawiam konkretne algorytmy i techniki uczenia maszynowego, które zostały użyte lub badane w  projekcie. Każda podsekcja jest poświęcona innemu algorytmowi lub technice. 

\subsection{Regresja liniowa}
Regresja liniowa to technika uczenia maszynowego wykorzystywana do modelowania zależności między jedną lub wieloma zmiennymi niezależnymi a zmienną zależną, która jest ciągła. Ta sekcja opisuje, w jaki sposób działa regresja liniowa, jakie są jej zalety i ograniczenia oraz w jakich sytuacjach może być użyteczna w analizie danych.

\subsection{Drzewa decyzyjne}
 Drzewa decyzyjne są algorytmem, który pozwala na podejmowanie decyzji w oparciu o serię pytań i warunków logicznych. Ta sekcja wyjaśnia, jak drzewa decyzyjne są budowane, jakie są ich cechy i jak mogą być używane 

\subsection{Sieci neuronowe}
Sieci neuronowe są algorytmami inspirowanymi budową mózgu, wykorzystywanymi do uczenia maszynowego. Ta sekcja opisuje podstawy sieci neuronowych, jakie są rodzaje sieci, jakie są zastosowania sieci neuronowych w problemach klasyfikacji i przewidywania.


\subsection{Maszyny wektorów nośnych}
Maszyny wektorów nośnych (SVM) to technika uczenia maszynowego, która jest wykorzystywana w zadaniach klasyfikacji i regresji. Wyjaśniasz, jak SVM działa, jakie są jego zalety, a także w jakich sytuacjach może być używan

\subsection{Algorytmy klastrowania}
Algorytmy klastrowania to techniki używane do grupowania danych na podstawie ich podobieństwa. Ta sekcja opisuje różne algorytmy klastrowania, jakie są ich zastosowania i jak mogą pomóc w analizie danych.

