\newpage % Rozdziały zaczynamy od nowej strony.
\section{Studium przypadku}
przedstawiam konkretne przypadki lub scenariusze, w których zastosowano algorytmy uczenia maszynowego w kontekście zarządzania łańcuchem dostaw. Każda podsekcja jest poświęcona innemu przypadkowi z wykorzystaniem tych algorytmów. 

\subsection{Studium przypadku 1: Optymalizacja zapasów}
W tej sekcji przedstawiam konkretny scenariusz związany z zarządzaniem łańcuchem dostaw, gdzie wykorzystano algorytmy uczenia maszynowego do optymalizacji poziomu zapasów. Opisuje, jakie były cele tego studium przypadku, jakie dane zostały użyte do analizy oraz jakie algorytmy były stosowane w celu zoptymalizowania zarządzania zapasami.

\subsection{Studium przypadku 2: Prognozowanie popytu}
prezentuje konkretny przykład zastosowania algorytmów uczenia maszynowego do prognozowania popytu na produkty lub usługi w łańcuchu dostaw. Wyjaśniam, jakie były cele prognozowania popytu, jakie dane zostały użyte do tworzenia modeli prognozowania i jakie wyniki uzyskano w analizie.

\subsection{Studium przypadku 3: Wybór dostawcy}
Ta sekcja opisuje sytuację, w której algorytmy uczenia maszynowego zostały zastosowane do wyboru dostawcy w łańcuchu dostaw. Omawiam, jakie czynniki i kryteria były brane pod uwagę podczas procesu wyboru dostawcy, jakie dane były używane do analizy i jak algorytmy wspomagały w podjęciu decyzji.


\subsection{Studium przypadku 4: Planowanie produkcji}
W tej sekcji prezentuję przykład zastosowania algorytmów uczenia maszynowego do planowania produkcji w ramach zarządzania łańcuchem dostaw. Opisuję cele planowania produkcji, używane dane, a także wykorzystane algorytmy i techniki do optymalizacji procesów produkcyjnych.

