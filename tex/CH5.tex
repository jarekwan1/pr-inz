\newpage % Rozdziały zaczynamy od nowej strony.
\section{Wyniki i dyskusja}
 prezentuje wyniki badań i analizuje ich znaczenie w kontekście zarządzania łańcuchem dostaw. 

\subsection{Porównanie wydajności algorytmów}
W tej sekcji dokonuje porównania wyników osiągniętych przez różne algorytmy uczenia maszynowego, które zostały użyte w Twoich studiach przypadku (lub w planowaniu produkcji). To pozwala na ocenę, który z algorytmów sprawdził się najlepiej w konkretnej sytuacji. Moge użyć różnych metryk oceny, aby porównać wydajność

\subsection{Interpretacja wyników}
W tej podsekcji analizuje i interpretuje wyniki uzyskane w badaniach przypadku . Wyjaśniasz, co otrzymane wyniki oznaczają w kontekście Twojego badania. Czy wyniki potwierdzają lub kwestionują pierwotne hipotezy lub cele? W jaki sposób algorytmy uczenia maszynowego wpłynęły na efektywność zarządzania łańcuchem dostaw lub planowania produkcji?

\subsection{Implikacje dla zarządzania łańcuchem dostaw}
W tej sekcji rozważam, jakie praktyczne implikacje wyników badań mają dla zarządzania łańcuchem dostaw. Jakie wnioski można wyciągnąć na temat optymalizacji procesów, efektywności czy podejmowania decyzji w łańcuchu dostaw na podstawie uzyskanych wyników? Jakie praktyczne zastosowania lub zalecenia można sformułować na podstawie  badań?

