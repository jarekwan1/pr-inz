\newpage % Rozdziały zaczynamy od nowej strony.
\section{Wstęp}




\subsection{Tło}

Współczesny świat (postepujaca globalizacja) charakteryzuje się dynamicznymi zmianami, rosnącą konkurencją oraz rosnącym zapotrzebowaniem na coraz bardzo skomplikowane usługi i produkty, stawia to przed przedsiębiorstwami wiele wyzwań związanych z efektywnym zarządzaniem łańcuchem dostaw. 

W szczególności kluczowe technologie mają ogromne znaczenie w długoterminowej perspektywie konkurencyjności firm. Dlatego warto rozważyć ich zastosowanie. Współczesne kluczowe technologie XXI wieku znane są jako sztuczna inteligencja (AI). Analogicznie do ludzkiego poznania, termin "sztuczna inteligencja" obejmuje systemy zdolne do przetwarzania informacji, rozumienia języka, wykonywania działań i skoncentrowania się na celach, a także zdobywania wiedzy do rozwiązywania problemów, przy wykorzystaniu uczenia maszynowego. Aby wykorzystać tę zdolność uczenia się przez system, uczenie maszynowe (ML) wyłoniło się jako niezależna dziedzina badań i technologii w kontekście sztucznej inteligencji. W odróżnieniu od ręcznego kodowania pojedynczych rozwiązań, takich jak korzystanie z reguł czy ontologii, wiedza z zakresu uczenia maszynowego jest automatycznie dostarczana przez odpowiednie systemy, które wykorzystują algorytmy oparte na empirycznych danych do rozwiązywania problemów.\cite{Weinke2023}


 Sztuczna inteligencja jest obecnie jedna z najszybciej rozwijających sie technik, mającą w sobie potencjał do przełomowego wpływu na sposób organizacji i funkcjonowania zarówno społeczeństwa jako całości, jak i pojedynczych osób.Juz obecnie widzimy oddziaływanie SI na gospodarke, w szczególnosci na produkcje przemyslową, gdzie tzw. przemysł 4.0, oparty na szerokiej robotyzacji zastepuje tradycyjne formy wytwarzania produktów.\cite{Nowak2022}


\subsection{Opis problemu}
   Dawniej począwszy od kadry kierowniczej wyższego szczebla po kadrę kierowniczą pierwszej linii,  mierzac się ze złożonymi i krytycznymi decyzjami to tradycyjnie takie decyzje zależały od ich doświadczenia i osądu.Jednakże rynek przestawił się na krótkie serie produkcyjne, aby zaspokoić szybko zmieniający się popyt, a koszty zostały obniżone na rzecz metod produkcji „dokładnie na czas”. decyzje stały się bardziej złożone. Jednocześnie procesy operacyjne (produkcja) staje sie bardziej zautomatyzowana i zintegrowana, co umożliwi większą kontrolę nad łańcuchem dostaw. Dlatego ostatnio coraz większą uwagę poświęca się technikom sztucznej inteligencji (AI).\cite{Wong2013}
   
   W dobie rozwijającej się i powszechnie dostępnej technologii oraz 
dostępu do aktualnej informacji nie wystarczy oferować wyroby i usługi w tzw. ,,standardzie”. 
Wyróżniać się wśród konkurencji, to znaczy działać nieszablonowo, oferować wyroby i usługi 
dedykowane, ,,szyte na miarę” pod indywidualne potrzeby klientów, przy zachowaniu 
wymaganego poziomu jakości. \cite{Jozwiak2017}

    Łańcuch dostaw to obszar istotny , który ma wpływ na  konkurencyjność, elastyczność i zdolność do czestego i szybkiego dostosowywania się do coraz szybciej zmieniającego sie rynku. W ten obszar doskonale wkomponuja sie  technologie z zakresu uczenia maszynowego które staja się znaczącym narzędziem wspierającym lub całkowicie decydujacym w zarządzaniu łańcuchem dostaw.W dzisiejszym środowisku biznesowym, które cechuje się rosnącą ilością dostępnych danych oraz coraz bardziej złożonymi danymi, algorytmy uczenia maszynowego zaczynają stanowić kluczowy element umożliwiający  podejmowanie bardziej poinformowanych decyzji, redukcję kosztów oraz zwiększenie efektywności operacyjnej.
    
    Biorąc pod uwagę złożoność zadań planowania, zarządzania i kontroli w przemysłowych łańcuchach wartości, aplikacje ML uznawane są za niezwykle istotne dla wspierania i autonomicznej realizacji logistycznych procesów decyzyjnych\cite{Weinke2023}
    
\subsection{Cele}
Celem jest zbadanie i ocena potencjału zastosowania wybranych algorytmów uczenia maszynowego w zarządzaniu łańcuchem dostaw. 
\subsection{Zakres badania}
Praca skupia się na analizie, projektowaniu oraz implementacji algorytmów uczenia maszynowego w różnych częsciach zarządzania łańcuchem dostaw, takich jak prognozowanie popytu, planowanie produkcji, zarządzanie zapasami .
