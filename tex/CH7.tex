\newpage % Rozdziały zaczynamy od nowej strony.
\section{Wniosek}
zakończenie pracy inżynierskiej i ma na celu podsumowanie kluczowych aspektów badania oraz przedstawienie ogólnych wniosków .

\subsection{Podsumowanie ustaleń}
W pierwszej części tej sekcji dokonuje podsumowania głównych ustaleń i wyników, które uzyskałem w trakcie swojego badania. To jest miejsce, gdzie można znaleźć najważniejsze informacje na temat  badania w jednym miejscu. Staram się podkreślić, co udało Ci się osiągnąć i jakie kluczowe wyniki uzyskałeś.

\subsection{Wkład}
 opisuje, jaki konkretny wkład praca wnosi do zarządzania łańcuchem dostaw i zastosowań uczenia maszynowego w SCM. Wyjaśniam, w jaki sposób badania rozwiązują istniejące problemy, rozwijają wiedzę lub otwierają nowe możliwości badawcze w tej dziedzinie. To jest miejsce, gdzie podkreślasz znaczenie pracy.

\subsection{Zastosowania praktyczne}
Omawiam praktyczne zastosowania wyników  badania w rzeczywistym środowisku biznesowym. Wyjaśniam, w jaki sposób przedsiębiorstwa lub organizacje mogą wykorzystać  ustalenia i techniki w praktyce. To pozwala  zrozumieć, jakie korzyści mogą wyniknąć z mojej pracy dla przemysłu i sektora SCM.

\subsection{Wnioski i uwagi końcowe}
W tej ostatniej części pracy podkreślam główne wnioski i formułuje uwagi końcowe na temat całego procesu badawczego. podziękowania za wsparcie, podsumowania, dlaczego ta praca jest istotna i jakie znaczenie ma dla dziedziny zarządzania łańcuchem dostaw oraz zastosowań uczenia maszynowego w SCM.