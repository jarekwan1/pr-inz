\newpage % Rozdziały zaczynamy od nowej strony.
\section{Wyzwania i przyszłe kierunki}
 analizuje istniejące wyzwania i przyszłe kierunki w dziedzinie zarządzania łańcuchem dostaw (Supply Chain Management - SCM) oraz wykorzystania technik uczenia maszynowego (ML) w tej dziedzinie. 

\subsection{Wyzwania we wdrażaniu ML w SCM}
W tej części pracy koncentruje się na identyfikacji i analizie wyzwań związanych z wdrażaniem technik uczenia maszynowego w zarządzaniu łańcuchem dostaw. Omawiasz trudności związane z gromadzeniem, przetwarzaniem i analizą dużych zbiorów danych, akceptacją przez organizację, kosztami wdrożenia, a także możliwymi problemami związanych z infrastrukturą i kompatybilnością z istniejącymi systemami SCM. Dostarczam także sugestii i rozważań dotyczących strategii radzenia sobie z tymi wyzwaniami.

\subsection{Przyszłe kierunki badań}
W tej części pracy rozważam przyszłe kierunki badań w dziedzinie zarządzania łańcuchem dostaw i zastosowań ML w SCM. Wskazujesz obszary, które mogą stać się głównymi tematami badawczymi w przyszłości, takie jak rozwijanie bardziej zaawansowanych modeli predykcyjnych, wykorzystanie danych czasu rzeczywistego, automatyzacja procesów SCM, czy też bardziej zaawansowane technologie śledzenia i monitorowania. Twoje refleksje mają na celu ukierunkowanie przyszłych prac badawczych i innowacji w tej dziedzinie.

\subsection{Względy etyczne i dotyczące prywatności}
 W tej sekcji zajmujesz się kwestiami etycznymi i ochroną prywatności w kontekście wykorzystania ML w SCM. Rozważam potencjalne zagrożenia związane z danymi, takie jak naruszenia prywatności klientów lub pracowników oraz możliwe skutki uboczne stosowania algorytmów uczenia maszynowego. Omawiam także istniejące lub proponowane środki ostrożności i regulacje, które można wprowadzić w celu ochrony danych i zachowania etycznych standardów w zarządzaniu łańcuchem dostaw.